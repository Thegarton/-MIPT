\documentclass[a4paper,12pt]{article} % тип документа
\usepackage[text={180mm, 260mm}, left=15mm, right=15mm]{geometry}
\usepackage{graphicx}
\usepackage{wrapfig}
\usepackage{hyperref}
\usepackage[rgb]{xcolor}	
\usepackage[utf8]{inputenc}	
\usepackage[english,russian]{babel}	
\usepackage{amsmath,amsfonts,amssymb,amsthm,mathtools} 
\usepackage{float} 
\usepackage{wasysym}
\usepackage{fourier} 
\usepackage{array}
\usepackage{soul}
\usepackage{makecell}
\usepackage{fontspec}
\usepackage{indentfirst}
\usepackage{caption}
\usepackage{subcaption}
\setmainfont{NewComputerModern}					
\usepackage{mathtext} 		
\usepackage{pst-node}% http://ctan.org/pkg/pst-node						
\usepackage{multirow}
\hypersetup{colorlinks=true,urlcolor=blue}
\usepackage[rgb]{xcolor}
\usepackage{icomma} 
\usepackage{euscript}
\usepackage{mathrsfs}
\usepackage{enumerate}
\usepackage{graphicx}
\usepackage{booktabs}
\usepackage{cmap}									
\usepackage{tabularx}
\usepackage{tikz}
\usepackage{longtable}
\title{
	Степанов Линейная алгебра.
}
\usepackage{easybmat}		
\begin{document}
	\maketitle
	\tableofcontents
	\section{02.02.24 1 лекция}
	\subsection{Ранг Матрицы}
	
	\noindent A $\in$ $M_{m,n}$($\mathbb{R}$).
	Строки столбы матрицы могут быть ЛЗ, ЛНЗ.\\
	
	\noindent $a_1$,$\dotso$,$a_m$ строки ЛЗ $\iff$ $\exists$ $\lambda_1$,$\dotso$,$\lambda_m$ $\in$ $\mathbb{R}$ $ \lambda_1^2 + \dotsb + \lambda_m^2 \neq 0 $ \\
	$\lambda_1 a_1 + \dotsb + \lambda_m a_m = 0 = \overbrace{(0,\dotso,0)}^{m} $\\
	
	\noindent$r_1(A)$ - строчный ранг матрицы A - максимальное количество ЛНЗ строк матрицы A.\\
	0 $\leq$ $r_1(A) \leq m$ \\
	
	\noindent$r_2(A)$ - столбчатый ранг матрицы A - максимальное количество ЛНЗ столбцов матрицы A. 
	0 $\leq$ $r_1(A) \leq n$\\
	\subsection{Минор Матрицы}
	\textbf{Определение 1.} Минором порядка k, где k $1\leq k \leq min\{m,n\}$ называется определителем матрицы,образованной элементами стоящими на пересечение некоторых выбранных k строк и k столбцов матрицы A. \\
	$1 \leq i_1 < i_2 < \dotso < i_k \leq m $\\
	$1 \leq j_1 < j_2 < \dotso < j_k \leq n $\\
	
	$M_{i_1 \dotso i_k}^{j_1 \dotso j_k}$ -- минор, обрзованный строками с номерами $i_1, \dotso, i_k$ и столбцами $j_1,\dotso, j_k$\\
	
	\noindent\textbf{Пример}\\
\[A_{3,4}=
\left(
\begin{BMAT}{cccc}{ccc}
	0 & 2 & 1 & 3 \\
	4 & 5 & 0 & -1 \\
	2 & 2 & 1 & 1 
	\addpath{(0,0,1)rrrrulllld}
	\addpath{(0,2,1)rrrrulllld}
	\addpath{(1,3,1)rdddluuu}
	\addpath{(3,3,1)rdddluuu}
\end{BMAT}
\right)
\]
	$M^4_2 = a_{24} = -1 \quad \quad \quad \quad M_{13}^{24}=\begin{vmatrix}
		2 & 3 \\
		2 & 1
	\end{vmatrix}=-4$\\
	Ранг матрицы A, r(A) = максимальный порядок не нулевого минора матрицы A.
	$r(A) = max\{0 \leq k \leq min\{m,n\}\} | \exists$ ненулевой минор порядка k в A, но миноры большого порядка либо $\nexists$ либо все не равны 0.\\
	\noindent\textbf{Пример}\\
	$	M_{123}^{123}=\begin{pmatrix}
		$0$ & $2$ & $1$\\
		4& $5$ & 0  \\
		$2$ & $2$ & $1$
	\end{pmatrix} \neq 0 \quad \Rightarrow \quad r(A) = 3 $
	\subsection{Теорема об инвариантности ранга при элементарных преобразованиях}
	Если матрица $A'$ получена из матрицы $A$ последовательностью элементарных преобразований строк и столбцов, то $r_1(A')=r_1(A)$ и $r_2(A') = r_2(A)$\\
	\textit{Доказательство после теории размерности векторных пространств}.\\
	
	\textbf{Следствие 1.} 
	$\forall$ матрицы A $r_1(A) = r_2(A)$\\
	\textbf{Доказательство:} Приведем матрицу A к ступенчатому виду   
		\[A'=
		\left(
		\begin{BMAT}{cccccc}{cccccc}
			0 & \dotso & a_{1j_{1}} & \dotso & a_{1j_{n-1}} & a_{1j_{n}}\\
			0 & \dotso & 0 & a_{2j_{2}} & \dotso & a_{2j_{n}} \\
			\vdots &  & \vdots & \vdots & \ddots & \vdots\\
			0 & \dotso & 0 & 0 & 0 & a_{rj_{r}}\\
			0 & \dotso & 0 & 0 & 0 & 0\\
			0 & \dotso & 0 & 0 & 0 & 0
			\addpath{(2,6,1)drdrdrdr}
		\end{BMAT}
		\right) \quad 1 \leq j_1 < j_2 < \dotso < j_r \leq n \qquad a_{1j_1},\dotso ,a_{rj_r} \neq 0\] 
	  Далее поделим k-ую строку на $a_{kj_k}$ и переместим столбы $j_1, \dotso, j_r$ в начало матрицы.
	  
	  $$
	  A' \Rightarrow A'' = \left(
	  \begin{BMAT}{cccc}{cccc}
	  	1 & a_{12}'' & \dotso & a_{1n}''\\
	  	0 & 1 & a_{23}'' & \dotso \\
	  	\vdots & \vdots & \ddots & \\
	  	0 & 0 & \dotso & 1\\
	  	%\addpath{(2,6,1)drdrdrdr}
	  \end{BMAT}
	  \right)
	  $$
	  Из второго столбца вычтем первый с коэффициентом $a_{12}'' \dotso$  из n-го 1-й с коэффициентом $a_{1n}''$ и т.д для 2-го столбца и 2 строки и т.д. 
	  \[
	  A\sim A''' = 
	  \left( \begin{array}{c|c}
	  	E_r & 0 \\
	  	\midrule
	  	0 & 0 \\
	  \end{array}\right)
	  \]
	  $из Teop_1 \Rightarrow r_i(A) = r_i(A'''), i = 1,2$\\
	  Первые $r$ строк матрицы $A'''$:\\
	  $e_1 = (1,0,\dotso,0, \dotso,0)$\\
	  $e_2 = (0,1,\dotso, 0, \dotso,0)$\\
	  $ \qquad \vdots $\\
	  $e_r = (0,0,\dotso,1,\dotso,0)$\\ 
	  
	  Если $\Sigma \lambda_i e_i = 0 = (0,\dotso, 0) \Rightarrow (\lambda_1, \dotso, \lambda_r, \dotso,0) \Rightarrow \lambda_1 = \dotsb = \lambda_r = 0 \Rightarrow $ Эти строки ЛНЗ $\Rightarrow r_1(A''') = r$.\\
	   Аналогично $r_2(A''') = r \Rightarrow r_1(A) = r_2(A)$ Ч.Т.Д.\\
	   
	   \textbf{Замечание}  
	   Вычисление ранга матрицы методом элементарных преобразований: Нужно привести матрицу А к ступенчатому виду путём преобразований строк.
	   Тогда $r_1(A) = r_2(A) = r$ -- количеству ненулевых строк в ступенчатом виде.\\
	   
	   \textbf{Определение 2.} Пусть M -- некоторый минор матрицы A минор $M'$ матрицы А называется окаймляющим для М если $M'$ получается из М добавлением одной строки и одного столбца.\\
	   Пример:
	   $$
	   A=\begin{pmatrix}
	   	1 & 2 & 3 & 4 \\
	   	0 & 0 & 1 & 1 \\
	   	2 & -1 & 3 & 2
	   \end{pmatrix} \quad M=M_{13}^{13}=\begin{vmatrix}
	   1 & 3 \\
	   2 & 3
	   \end{vmatrix}=-3
	   $$
	   Окаймляющие: $M'^{123}_{123}, \ M'^{134}_{123}$\\
	   
	   \textbf{Определение 3} 
	   Минор M матрицы А называется базисным если M $\neq$ 0, а все его окаймляющие миноры либо $\nexists$, либо равны 0. Строки и столбцы входящие в базисные миноры называются базисными.\\
	   
	   \subsection{Теорема о базисном миноре}
		Базисные строки (столбцы) любой матрицы ЛНЗ. Остальные строки(столбцы) линейно выражаются через  базисные.\\
		\textbf{Доказательство:} Для строк, для столбцов аналогично.
		Если базисные строки ЛЗ, то и строки базисного минора ЛЗ $\Rightarrow$ он равен 0 -- противоречие.\\
		Будем считать, что базисный	 минор М = $M_{1,\dotso,r}^{1,\dotso,r}$
		Рассмотрим определители $M'$, которые получаются добавлением к M i-й строки,  i > r и $\forall$ столбцов матрицы А.\\
		Если мы добавим j-й столбце с $j \leq r$ то в  $M'$два одинаковых столбца $\Rightarrow M' = 0$.\\
		Если $j > r$, то $M'$ -- окаймлённный минор для М $\Rightarrow M' = 0$
		$$
		0 = M' =\begin{vmatrix}
			a_{11} & \dotso & a_{1r} & a_{1j} \\
			\vdots & \ddots &\vdots & \vdots\\
			a_{r1} & \dotso & a_{rr} & a_{rj}\\
			a_{i1} & \dotso & a_{ir} & a_{ij}
		\end{vmatrix} = \text{(Разложим по последнему столбцу)}=A'_{1j}a_1j
 + A'_{2j}a{2j} + \dotsb + A'_{rj}{rj}+Ma_{ij} = 0	  $$
 \begin{equation}
 	\label{1}
 	a_{ij} = -\dfrac{A_{1j}'}{M}a_{1j} - \dotso - \dfrac{A_{rj}'}{M}a_{rj} \quad \forall j = 1,\dotso ,n
 \end{equation}
 Формула \eqref{1} выражает элемент i - й строки через соответствующие элементы 1-й, ... , r-й строк. Коэффициенты не зависят от j $\Rightarrow$ i-я строка линейно выражается через базисные строки. ЧТД.\\
 
 
 \textbf{Следствие 2(Теорема о ранге матрицы)} 
$\forall$ матриц А $r_1(A) = r_2(A) = r(A)$ и равны порядку любого базисного минора. 

\textbf{Доказательство}
$r_1(A) = r_2(A)$ -- уже доказано. Пусть M -- базисный минор матрицы A. $M_{12\dotso r}^{12\dotso r}$. (r+1)--я,$\dotso$,m--я строки -- линейная комбинация базисных строк.
$$
A\sim\begin{pmatrix}
	a_{11} &  a_{12} & \dotso & a_{1n} \\
	\vdots & \vdots &\ddots & \vdots\\
	a_{r1} & a_{r2}& \dotso & a_{rn}\\
	0 & 0 &\dotso & 0\\
	0 & 0 &\dotso & 0
\end{pmatrix} = A' \quad \text{из теоремы 1} \Rightarrow r_1(A) = r_1(A')
$$
Строки (1),$\dotso$,(r) ЛНЗ $\Rightarrow r_1(A') = r$. $r$ -- порядок базисного минора. Если $M$ -- максимальный по порядку минор $\neq$ 0, то он автоматически базисный $\Rightarrow r(A) = r = r_1(A) = r_2(A)$

\textbf{Определение 4}
Рангом матрицы А называется число rkA определённое любым выше указанным эквивалентным способом.
	\end{document}
		